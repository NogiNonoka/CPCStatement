\section{C}
\begin{frame}
\frametitle{C - Superdog's Kindergarten Mathematics}
难度:Medium-hard
类型:数学

这个式子有非常好的对称性,所以可以进行换元处理。
\begin{block}{推导过程}
为了方便书写,这里令 $a+b=2m$。

令$a=m-t,b=m+t$

得$\frac{1}{(m-t)^2+n}+\frac{1}{(m+t)^2+n}$

通分得$\frac{t^2+m^2+n}{(t^2+m^2+n)-(2mt)^2}$

再换元,令$k=t^2+m^2+n$,$ m^2+n <k \leq 2m^2+n$

得$\frac{2k}{k^2-4m^2(k-m^2-n)}$

化简得$\frac{2k}{k^2-4m^2k+4m^2(m^2+n)}$

上下同除k得$\frac{2}{k+\frac{4m^2(m^2+n)}{k}-4m^2}$
\end{block}
\end{frame}

\begin{frame}

\frametitle{C - Superdog's Kindergarten Mathematics}
\begin{block}{推导过程}
用基本不等式得到最大值为$\frac{2}{4m\sqrt{m^2+n}-4m^2}$,$k=2m\sqrt{m^2+n}$时成立。

不存在$k=2m\sqrt{m^2+n}>2m^2+n$的情况,两边平方后非常好比较不做证明。

存在$k=2m\sqrt{m^2+n}<m^2+n$的情况,此时$n \leq 3m^2$,$k=m^2+n$,即$t=0$,也就是$a=b$,最大值$\frac{2}{m^2+n}$
\end{block}
\end{frame}