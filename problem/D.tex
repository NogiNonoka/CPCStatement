\newpage
\section{Problem D 提瓦特初等元素论}
{ \limitfont{}
Input file: standard input \par
Output file: standard output \par
Time limit: 1000ms \par
Memory limit: 512MB \par
}
\subsection*{题目描述}

众所周知,在提瓦特大陆上除去草元素(Dendro)外有6种元素,他们分别是:火(Pyro)、水( Hydro)、风( Anemo)、雷(Electro)、冰(Cryo)、岩(Geo)。他们之间两两会发生反应:

水和冰相遇发生冻结反应(Frozen)

火和水相遇会发生蒸发反应(Vaporize)

水和雷相遇会发生感电反应(Electro-Charged)

火和雷相遇会发生超载反应(Overloaded)

冰和雷相遇会发生超导反应(Superconduct)

冰和火相遇会发生融化反应(Melt)

风元素和除岩元素外的其他元素会发生扩散反应(Swirl)

岩元素和除风元素外的其他元素会发生结晶反应(Crystallize).

提瓦特大陆上元素攻击是一种强力的攻击方式,因此东走夕对元素反应非常感兴趣。他发现提瓦特大陆上的元素攻击不但元素类型不同,其中包含的**元素量**也不同。为了深入研究提瓦特元素论,东走夕找了一只落单的丘丘王,并对它施加了各种各样的元素攻击。东走夕在暴打丘丘王时发现了以下规律:

1、风元素和岩元素不会产生元素附着。

2、如果丘丘王身上没有元素附着,那么元素攻击会让丘丘王带上和此次元素攻击同类型且等元素量的元素。

3、如果丘丘王身上已经有元素附着,那么不同类型元素攻击会在丘丘王的身上发生剧烈的反应。反应时会消耗丘丘王身上的元素附着量,值等于此次元素攻击的元素量。但如果此次攻击的元素量大于丘丘王身上的元素附着量,只会消耗完元素附着量但不会产生新的元素附着。

4、如果丘丘王身上已经有元素附着,那么同类型元素攻击会刷新丘丘王身上的元素附着,新的元素附着量为元素攻击和已有元素附着量的较大值。



东走夕被元素反应深深吸引,以至于忘记了记录发生了什么反应,所幸他记得自己释放的每一次元素攻击的类型和元素量。现在,东走夕想请你告诉他发生了什么反应。

\subsection*{输入描述}

第一行一个正整数 $n$。

接下来 $n$ 行,每行包括一个字符串 $s_i$ 和正整数 $a_i$,表示元素攻击的类型和元素量,中间用一个空格隔开。

\subsection*{输出描述}

每行一个字符串,表示发生的反应类型。

\subsection*{测试样例}

\begin{table}[H]
\begin{tabularx}{\textwidth}{|X|X|}
    \hline
    \textbf{Standard Input} & \textbf{Standard Output} \\ 
    \hline 
    \tablecell{
        10 \\ Electro 6 \\ Cryo 9 \\ Pyro 5 \\ Anemo 1 \\ Pyro 1 \\ Pyro 4 \\ Cryo 2 \\ Anemo 1 \\ Geo 6 \\ Hydro 1 \\
    } & 
    \tablecell{Superconduct \\ Swirl \\ Melt \\ Swirl \\Crystallize \\ \\ \\ \\ \\ \\ \\} \\
    \hline
\end{tabularx}
\end{table}

\subsection*{样例解释}

第一次攻击,将丘丘王附上雷元素,元素附着量为 $6$。

第二次攻击,丘丘王身上的雷元素和冰元素反应,发生超导反应,此时丘丘王身上无元素附着。

第三次攻击,将丘丘王附上火元素,元素附着量为 $5$。

第四次攻击,丘丘王身上的火元素和风元素反应,发生扩散反应,此时丘丘王身上还剩余 $4$ 元素量的火元素附着。

第五、六次攻击,刷新丘丘王身上的火元素附着,火元素附着量为 $4$。

第七次攻击,丘丘王身上的火元素和冰元素反应,发生融化反应,剩余 $2$ 元素量的火元素附着。

第八次攻击,丘丘王身上的火元素和风元素反应,发生扩散反应,剩余 $1$ 元素量的火元素附着。

第九次攻击,丘丘王身上的火元素和风元素反应,发生结晶反应,此时丘丘王身上无元素附着。

第十次攻击,将丘丘王附上水元素,元素附着量为 $1$。

\subsection*{数据规模}

对于$20 \%$的数据:$1 \leq n \leq 10,1 \leq a_i \leq 10$

对于$50 \%$的数据:$1 \leq n \leq 1000,1 \leq a_i \leq 100$

对于$100 \%$的数据:$1 \leq n \leq 100000,1 \leq a_i \leq 1000$

