\newpage
\section{Problem E 共鸣}
{ \limitfont{}
Input file: standard input \par
Output file: standard output \par
Time limit: 1000ms \par
Memory limit: 512MB \par
}
\subsection*{题目描述}

在提瓦特世界中,存在着一种很特殊的元素现象-岩元素共鸣。
岩元素共鸣通常出现在岩造物之间。

套餐用户在南天门的伏龙树下发现了一个奇怪的封印,一阵阴风袭来,套餐用户被吸入封印中。套餐用户来到了一个奇特的地界,在此地,岩造物不断地被创造以及消散。经过一段时间的观察,套餐用户发现其中只有两种岩造物,特殊的岩造物以及普通的岩造物。

特殊的岩造物可以与同时与多个普通的岩造物产生共鸣,同样的,普通岩造物可以同时与多个特殊岩造物产生共鸣。特殊的岩造物之间不能产生共鸣现象,同样的,普通的岩造物之间也不会产生共鸣现象。

岩造物与岩造物之间每秒产生一次共鸣,一个特殊岩造物与多个普通岩造物可以同时产生多次共鸣。特殊岩造物与普通岩造物之间的共鸣无视距离。岩造物刚产生的那一刻会产生共鸣,岩造物毁灭的那一刻不会产生共鸣。

在这个地界中,岩造物不断地生成或毁灭,没有一个岩造物会永久存在。套餐用户在这个地界中呆了很久,四处游行。他发现只需要从某一秒(这一秒没有产生岩元素共鸣)开始,到某一秒(这一秒恰好也没有产生岩元素共鸣)结束,数出这一段时间内总共产生了多少次岩元素共鸣并且将其刻在石板上,如果答案正确,即可离开此地。套餐用户分析完后就立刻开始计数,我们将开始时刻设置为时刻1,即从时刻1开始考虑本问题。这对套餐用户来说似乎有点困难,套餐用户请你帮一帮他。

\subsection*{输入描述}

第一行输入一个整数 $N(1 \le N \le 1,000,000)$,表示接下来将会有 $N$ 个岩造物生成。

接下来的 $N$ 行,每一行输入三个正整数 $s_i(1 \le s_i \le 1,100,000),e_i(1 \le e_i \le 1,100,000),t_i(t_i \in \{0,1\})$,分别表示第 $i$ 个岩造物的被创造时间,毁灭时间,岩造物属性。其中,$t_i=1$ 表示第 $i$ 个岩造物为特殊岩造物,$t_i=0$ 表示第 $i$ 个岩造物为普通岩造物。保证 $e_i > s_i$。

输入数据保证必定有两个时刻,使得没有岩元素共鸣产生。

\subsection*{输出描述}

输出从第一次出现没有岩元素共鸣产生的时刻,到第二次出现没有岩元素共鸣产生的时刻之间,总共产生了多少次岩元素共鸣。

由于答案可能很大,请输出结果模 $19260817$ 后得出的答案。

\subsection*{测试样例}

\begin{table}[H]
\begin{tabularx}{\textwidth}{|X|X|}
    \hline
    \textbf{Standard Input} & \textbf{Standard Output} \\ 
    \hline
    \tablecell{
        5 \\
        1 2 1 \\
        1 2 0 \\
        3 10 1 \\
        3 10 0 \\
        4 10 1 \\
    } &
    \tablecell{
        13 \\ \\ \\ \\ \\ \\
    } \\
    \hline
\end{tabularx}
\end{table}

\subsection*{样例解释}

有5个岩造物将被创造。

第一个岩造物在时刻1被创造出来,在时刻2毁灭。是特殊岩造物

第二个岩造物在时刻1被创造出来,在时刻2毁灭。是普通岩造物

......

第五个岩造物在时刻4被创造出来,在时刻10毁灭。是特殊岩造物

根据数据可以得到每个时刻的特殊岩造物和普通岩造物的数量

时刻1:特殊1个,普通1个

时刻2:特殊0个,普通0个(因为在这个时刻毁灭了) [第一个时刻,使得没有共鸣产生]

时刻3:特殊1个,普通一个,一次共鸣

时刻4:特殊2个,普通一个,二次共鸣

时刻5:同上

时刻6:同上

时刻7:同上

时刻8:同上

时刻9:同上

时刻10:特殊0个,普通0个[第二个时刻,使得没有共鸣产生]

从时刻2开始到时刻10结束,一共产生 $1+2*6=13$ 次共鸣,所以输出13。

\subsection*{提示}

\begin{lstlisting}
int res = 100;
int ress = res % 3; // 这个就是模运算,即取除3的余数
cout << ress; // 输出1
\end{lstlisting}