\newpage
\section{Problem H \ \ wzy的伤害计算}
{ \limitfont{}
Input file: Standard Input \par
Output file: Standard Output \par
Time limit: 1000ms \par
Memory limit: 512MB \par
}
\subsection*{题目描述}

$wzy$最近沉迷于游戏《怪物猎人世界:冰原》。作为一名硬核玩家,他热衷于计算武器攻击的伤害。当自己计算的伤害与实际打出的相等时,他会相当的有成就感。
​首先,$wzy$要告诉你如何计算伤害,影响伤害的有以下因素:怪物肉质,武器状态,猎人装备,道具,动作值。

​动作值:可以理解为攻击动作的力量,是一个大于$0$实数。

怪物肉质:怪物肉质是一个位于区间$(0,1]$的实数,表明了怪物对伤害的吸收能力强弱,但实际情况下怪物的肉质是被软化的,设原先肉质为$X$,则软化后肉质为$X+(1-X)/4$。

​武器状态:每把武器都有一个武器面板,及武器基本倍率,将武器面板除以武器基本倍率可以得到武器基本伤害。

​道具:这些东西提供了攻击力固定数值加成,具体如下:

\begin{table}[H]
    \begin{tabularx}{\textwidth}{|X|X|}
    \hline
    \textbf{道具} & \textbf{攻击力固定数值加成} \\
    \hline
    {大鬼人药}         & {7 点攻击力} \\ \hline
    {鬼人粉尘}                  & {10 点攻击力}         \\ \hline
    {怪力种子 / 怪力药丸(同时只能使用一个)} & {10/25 点攻击力}      \\ \hline
    \end{tabularx}
\end{table}

​猎人装备:提供了第一类百分比加成,固定数值加成与第二类百分比加成,具体如下:

\begin{table}[H]
\begin{tabularx}{\textwidth}{|X|X|}
\hline
{ \textbf{固定数值加成}}   & { \textbf{只增加固定值}}                         \\ \hline
{ 攻击珠}               & { 每级提供 3 点攻击力,最高 7 级}                      \\ \hline
{ 挑战珠}               & { 每级提供 4 点攻击力,最高 7 级}                      \\ \hline
{ 无伤珠}               & { 1,2,3 级分别提供 5,10,20 点攻击力}                \\ \hline
{ 怨恨珠}               & { 每级提供 5 点攻击力,最高 5 级}                      \\ \hline
{ 猫的攻击珠}             & { 每级提供 5 点攻击力,最高 3 级}                      \\ \hline
{ \textbf{第一类百分比加成}} & { \textbf{见下方的计算公式}}                       \\ \hline
{ 猫的火场怪力}            & { 提供 1.35 倍攻击力加成}                          \\ \hline
{ 不屈}                & { 1,2 级分别提供 1.1 倍,1.2 倍攻击力加成}              \\ \hline
{ \textbf{第二类百分比加成}} & { \textbf{见下方的计算公式}}                       \\ \hline
{ 超会心}               & { 1,2,3,4 级分别提供 1.25,1.3,1.35,1.40 倍攻击力加成} \\ \hline
{ 斩味}                & { 白色提供 1.32 倍攻击力加成,紫色提供 1.39 倍攻击力加成}       \\ \hline
{ 寒气炼成}              & { 提供 1.3 倍攻击力加成}                           \\ \hline
\end{tabularx}
\end{table}

​		计算公式:

​		$\texttt{最终伤害}=EFR\times \texttt{动作值} \times \texttt{怪物肉质}$。

​		$EFR = (( \texttt{武器基本伤害}\times \texttt{第一类百分比})+ \texttt{固定数值})\times \texttt{第二类百分比}$。

\subsection*{输入描述}

输入共四行,第一行为$Dmg,lim,X,Y$,分别表示武器面板,武器基本倍率,怪物肉质及动作值。

第二行为$a_1,a_2,a_3$三个整数,分别表示了大鬼人药,鬼人粉尘,怪力种子/鬼人药丸的使用情况。其中$a_i=0$表示未使用,$a_i=1$代表使用,对于怪力种子/鬼人药丸,$a_3=0,1,2$分别代表了均未使用,使用了怪力种子,使用了鬼人药丸。

第三行为$b_1,b_2,b_3,b_4,b_5$五个整数,表示了攻击珠,挑战珠,无伤珠,怨恨珠,猫的攻击珠的使用情况。其中$b_i$代表有$b_i$级对应珠子。

第四行为$c_1,c_2,c_3,c_4,c_5$五个整数,表示了猫的火场怪力,不屈,超会心,寒气炼成,斩味使用情况。其中$c_i=0$代表未发动,$c_i=1$表示发动/发动一级。对于斩味,$c_5=1,2$分别表示了白斩和紫斩。保证$c_5=1$或$c_5=2$。

保证$Dmg \le 2^{30}$。

\subsection*{输出描述}

输出一行整数,表示怪物肉质软化后的最终伤害,因为怪物猎人世界里的伤害不存在小数,所以你只要将答案四舍五入并输出即可。

\subsection*{测试样例}

\begin{table}[H]
\begin{tabularx}{\textwidth}{|X|X|}
    \hline
    \textbf{Standard Input} & \textbf{Standard Output} \\ 
    \hline 
    \tablecell{
        500 2.0 1.0 2.5 \\
        0 1 2 \\
        6 5 0 4 2 \\
        1 0 2 0 2 \\
    } & 
    \tablecell{ 1990 \\ \\ \\ \\} \\
    \hline
\end{tabularx}
\end{table}
\subsection*{样例解释}
武器基本伤害为$500/2.0=250$

由于肉质为$1.0$,则软化后肉质依然为$1.0$。

第一类百分比为$1.35$(只发动了猫的火场怪力)

固定数值为$10+25+6\times3+5\times4+4\times5+2\times5=103$。(鬼人粉尘,鬼人药丸,$6$级攻击珠,$5$级挑战珠,$4$级怨恨珠,$2$级猫的攻击珠)

第二类百分比为$1.3\times1.39=1.807$($2$级超会心与紫斩)

故真实伤害为$((250\times 1.35)+103)\times 1.807\times 1.0\times 2.5=1989.95875$,四舍五入为$1990$。