\newpage
\section{Problem B \ \ 我有一段绝妙的代码,可惜这里写不下}
{ \limitfont{}
Input file: Standard Input \par
Output file: Standard Output \par
Time limit: 1000ms \par
Memory limit: 512MB \par
}
\subsection*{题目描述}

Superdog最近总是突发奇想,今天他想出了一道绝妙的题目,题目简化后如下:已知一个01序列,每一时刻,所有序列右移一位,最右端的值被舍弃,最左端补上指定某几位的异或和。

如,序列为“1000”,令最左端的值等于第$1,4$位的异或和。则:

第一次操作,序列变为“1000”,最左端的值为1\^{}0。

第二次操作,序列变为“1100”,最左端的值为1\^{}0。

第三次操作,序列变为“1110”,最左端的值为1\^{}0。

第四次操作,序列变为“1111”,最左端的值为1\^{}0。

第五次操作,序列变为“0111”,最左端的值为 1\^{}1。

......

很明显,这个序列必然有周期。且长度为$n$的序列周期最长为$2^n-1$。因为$n$位的01序列最多有$2^n$种排列,但是全0的情况显然无法成立,所以周期最长为$2^n-1$。

上述示例中的序列就是一个周期为$2^n-1$的序列。

那么,什么时候这个序列的周期可以取到最长呢?

Superdog稍许思索,有了思路。于是他在输出文件中写下了一串拉丁文:``Cuius rei demonstrationem mirabilem sane detexi.Hanc marginis exiguitas non caperet.''。

意为:关于此,我确信已发现了一种美妙的证法 ,可惜这里空白的地方太小,写不下。

\subsection*{输入描述}

Superdog觉得这个问题非常简单,所以没有输入。

\subsection*{输出描述}

Superdog忘记修改输出文件,所以输出文件就是上述的拉丁文。

\subsection*{测试样例}

\begin{table}[H]
    \begin{tabularx}{\textwidth}{|X|X|}
        \hline
        \textbf{Standard Output} \\
        \hline
         Cuius rei demonstrationem mirabilem sane detexi.Hanc marginis exiguitas non caperet. \\
        \hline
    \end{tabularx}
\end{table}

\subsection*{其他}
有兴趣可以思考一下上述问题qaq~
